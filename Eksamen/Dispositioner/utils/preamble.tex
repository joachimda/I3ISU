%============================================================%
%=== Document class and stuff ===============================%
%============================================================%
\documentclass[
10pt, % skriftstørrelse
a4paper, % papirtype
]{article}
%scrartcl

%============================================================%
%=== Packages ===============================================%
%============================================================%
\usepackage{blindtext} % til random tekst
\usepackage{lipsum} % bogus tekst
\usepackage{fancyhdr} % til header og footers
\usepackage[utf8]{inputenc} % æ ø å
%\usepackage{xcolor} % også til kode
\usepackage[nodayofweek]{datetime}
\newdateformat{mydate}{\twodigit{\THEDAY}{ }\shortmonthname[\THEMONTH], \THEYEAR}
\usepackage{graphicx}
\usepackage{nopageno}
\usepackage{amsmath}
\usepackage{caption}
\usepackage{subcaption}
\usepackage{wrapfig}
\usepackage[usenames,dvipsnames,table]{xcolor}
\usepackage{float}
\usepackage{tikz}
\usepackage[pages=some]{background}
\usepackage{lastpage}
\usepackage{tocloft} % needed for dotfill in contents
\usepackage{url}
\usepackage{hyperref}
\hypersetup{
	colorlinks,
	linkcolor={red!50!black}, % used to be: 'red!50!black', men det fuckede indholdsfortegnelsen...
	citecolor={blue!50!black},
	urlcolor={blue!80!black}
}

\usepackage[perpage]{footmisc} %% nulstiller footnote counter på hver side
\usepackage{todonotes} %% tillader "to-do"-liste vha. \todo{opgave}
\usepackage{datetime}
\usepackage{dblfnote} %% footnotes in two colums
\DFNalwaysdouble %% footnotes in two colums

%============================================================%
%=== Package settings========================================%
%============================================================%

%=== Papir størrelse og margin ==============================%
\usepackage { geometry }
\geometry {
	a4paper, 
	left 	= 3cm, 
	right 	= 3cm, 
	top 	= 4cm, 
	bottom 	= 4cm 
} 

%=== Indstillinger til kodestykker ==========================%
\usepackage{listings} % kode
\lstset{
	language = C++, 
	backgroundcolor=\color{black!5},
	basicstyle=\footnotesize,
	commentstyle=\color{mygreen},
	%frame=single, 						% laver en ramme om koden
	keepspaces=true, 					% beholde indrykning
	keywordstyle=\color{blue},
	numbers=left, 
	numbersep=5pt, 						% afstand fra nummer til kode
	numberstyle=\color{mygray}, 		% linjenummer farve
	stringstyle=\color{mymauve}, 		% string farve
	tabsize=5,
	captionpos=b, % sets the caption-position to bottom
	otherkeywords={Node, Graph, NULLpthread_t, pthread_create, pthread_join}, 	% til keywords 
}
\renewcommand\lstlistingname{Kode}
% === Farver til kode =======================================%
\definecolor{mygreen}{rgb}{0,0.6,0}
\definecolor{mygray}{rgb}{0.5,0.5,0.5}
\definecolor{mymauve}{rgb}{0.58,0,0.82}

% === Header og footer indstillinger ========================%
\pagestyle{fancy}	% sætter nyt headerformat i brug
\fancyhf{}			% kan ikke huske, men er sikkert vigtig

% === Indholdsfortegnelse ===================================%
\usepackage[english]{babel}	% bruges til at skrive ihft. på dansk
\addto\captionsenglish{\renewcommand{\contentsname}{Indholdsfortegnelse}} % skriver inholdfortegnelse på dansk

\addtolength{\jot}{1em} % gør rowspacing med align større

\newcommand{\derp}{\lipsum[1]}	% generere random tekst til test af format
