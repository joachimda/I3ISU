\section{OS API}

\subsection{Sub topics}

\begin{itemize}
	\item The design philosophy - Why OO and OS Api?
	\item Elaborate on the challenge of building it and its currenct design:
	\begin{itemize}
		\item The PIMPL / Cheshire Cat idiom - The how and why.
		\item CPU / OS Architecture.
	\end{itemize}
	\item Effect on design/implementation:
	\begin{itemize}
		\item MQs (Message queues) used with pthreads contra MQ used in OO OS Api.
		\item RAII in use.
		\item Using Threads before and now.
	\end{itemize}
	\item UML Diagrams to implementation (class and sequence) - How?
\end{itemize}

\subsection{Curriculum}

\begin{itemize}
	\item Slides: OS Api".
	\item OLA: OSAL SERNA SAC10".
	\item OLA: Speciffcation of an OS Api".
	\item Kerrisk: Chapter 35: Process Priorities and Schedul-ing".
\end{itemize}

\subsection{Exercises}

\begin{itemize}
	\item OS API.
\end{itemize}

\subsection{The design philosophy - Why OO and OS Api?}

\subsection{Elaborate on the challenge of building it and its currenct design}

\subsubsection{The PIMPL / Cheshire Cat idiom - The how and why}

\subsubsection{CPU / OS Architecture}

\subsection{Effect on design/implementation}

\subsubsection{MQs (Message queues) used with pthreads contra MQ used in OO OS Api}

\subsubsection{RAII in use}


\subsubsection{Using Threads before and now}
Før når vi skulle anvende tråde ville vi gøre noget i stil med det vist i kode udsnit~\ref{code:threads_before}, hvor en tråd erklæres, startes og blokere til den er færdig.

\begin{lstlisting}[caption=Anvendelse af tråde før OSAPI, label=code:threads_before, morekeywords={pthread_t, pthread_create, pthread_join}]
pthread_t thread;
pthread_create(&thread, NULL, func, (void*)&arg);
pthread_join(thread, NULL);
\end{lstlisting}

Med OSAPI er det hele mere Objekt Orienteret og generelt nemmere at bruge. Først skal en trådklasse laves (se kode udsnit~\ref{code:threadfunctor}) ved at arve fra \textit{ThreadFunctor} som skulle færdiggøres i øvelse 6.

\begin{lstlisting}[caption=Trådklasse via nedarvning fra ThreadFunctor, label=code:threadfunctor]
#include <osapi/Thread.hpp>

class ThreadClass : public ThreadFunctor 
{
public:
	virtual void run() 
	{
		// job of thread while running
	}
private:
	// class members
};
\end{lstlisting}

Kode udsnit~\ref{code:threads_after} viser hvordan vi nu kan bruge tråde ved hjælp af trådklassen i udsnit~\ref{code:threadfunctor} til let at starte en tråd.

\begin{lstlisting}[caption=Anvendelse af tråde efter OSAPI,
label=code:threads_after,
morekeywords={ThreadClass, Thread, join, start}]
ThreadClass threadclass();	// class to be run in thread
Thread thread(&threadclass);	// thread which the class will run in

thread.start();
thread.join();
\end{lstlisting}

\subsection{UML Diagrams to implementation (class and sequence) - How}

