\section{Thread communication}

\subsection{Sub topic}

\begin{itemize}
	\item The challenges performing intra-process communication.
	\item Message queue.
	\begin{itemize}
		\item The premises for designing it.
		\item Various design solutions - Which one chosen and why.
		\item Its design and implementation.
	\end{itemize}	
	\item Impact on design/implementation between before and after the Message Queue.
	\item Event Driven Programming.
	\begin{itemize}
		\item Basic idea.
		\item Reactiveness.
		\item Design - e.g. from sequence diagrams to code (or vice versa).
	\end{itemize}
\end{itemize}

\subsection{Curriculum}

\begin{itemize}
	\item Slides: "Inter-Thread Communication".
	\item OLA: "Event Driven Programming: Introduction, Tutorial, History - Pages 1-19 \& 30-51".
	\item OLA: "Programming with Threads - chapters 4 \& 6".
\end{itemize}

\subsection{Exercises}

\begin{itemize}
	\item Thread Communication
\end{itemize}

\newpage

\subsection{The challenges performing Intra-Process Communication}\label{sec:ipc}
Ulemper ved at implementere IPC med mutex\footnote{Mere om mutexes, se afsnit~\ref{sec:mutex}.} og conditional\footnote{Conditionals i samspil med mutexes, se afsnit~\ref{sec:mtxcond}.} variable, eller semephore\footnote{Semaphore, se afsnit~\ref{sec:semaphore}.}.

\begin{itemize}
	\item Det er en udfordring at undgå deadlocks\footnote{Mere i afsnit~\ref{sec:deadlock}.} og timing issues (race conditions).
	\item Programmer bliver svære at læse og gennemskue (readability).
	\item Hvis en tråd venter på en bestemt betingelse (fx. condition variable), så laver den ikke noget fornuftigt i mellemtiden.
	\item Hvis tråde deler ressourcer:
	\begin{itemize}
		\item Rækkefølgen hvori de tager/låser ressourcen skal planlægges grundigt for at undgå deadlocks.
		\item Der skal muligvis låses flere mutexes, hvilket kan være meget tidskrævende.
	\end{itemize}
\end{itemize}

Optimalt vil vi forsøge at lave en løsning hvor følgende gælder:

\begin{itemize}
	\item Processering i en tråd kræver ikke noget lås (mutex etc). 
	\item Tråde kan kommunikere med hinanden.
	\item Kommunikation mellem tråde skal kunne indeholde data.
	\item Tråde skal kunne sende disse beskeder når de vil. De skal altså ikke være bundet af noget bestemt flow i programmet.
	\item Mange tråde kan kommunikere med flere andre tråde.
\end{itemize}

\subsection{Message queue}
Disse beskedkøer skal fungere som ''mellemstation'' hvor besked kan opbevares før tråden er klar til at håndtere beskeden. Brugte vi ikke en sådan mekanisme ville beskeden ''gå tabt''. Alternativt skal vi bruge ressourcer på at synkronisere den sendende og modtagende part, således at beskeden sendes og modtages på ''samme tid''.

\subsubsection{The premises for designing it}
Producer/consumer problemet\footnote{Yderligere beskrevet i afsnit~\ref{sec:producer_consumer_probelm}} skal løses:
\begin{itemize}
	\item \textit{send()}
	putter en besked i køen og blokere hvis køen er fuld.
	
	\item \textit{receive()}
	henter en besked fra køen og blokere hvis køen er tom.
\end{itemize}
Her kommer conditionals ind i billedet. To conditions - QueueFull og QueueEmpty.

\subsubsection{Various design solutions - Which one chosen and why}
Implementeres med en FIFO kø af typen STL \textit{queue}, eventuelt med max på antal beskeder.

\paragraph{Beskedtypen}

Vi bruger i opgaven beskeder der arver fra en basisbesked. På denne måde er besked objekterne typesikre idet de kan slettes med deres \textbf{baseklasse pointer}.
Det er muligt at lave beskeder på andre måder:

\begin{itemize}
	\item void* - ingen typesikkerhed (delete problemer).
	
	\item Templates - For kompleks.
\end{itemize}

Vi caster vore basis message til den messagetype der ønskes.

\paragraph{Queue items}

De items som køerne indeholder. Kan eksempelvis implementeres med en struct, bestående af et id og et beskedobjekt af basisklassen Message.

Dispatcheren kan da switche på QueueItems id, og Handleren kan  reagere på beskeden.

\paragraph{Asynkrone køer}
\begin{itemize}
	\item Modtageren skal ikke have svar med det samme.
	\item I sekvensdiagrammet tegnes besked som en åben pil, se figur~\ref{fig:msq_seqdia}.
	\item Afsenderen kan forsætte operationer uden svar fra modtager.
\end{itemize}

\paragraph{Synkrone køer}
\begin{itemize}
	\item Modtageren skal have svar med det samme.
	\item I sekvensdiagrammet tegnes besked som en lukket pil, figur~\ref{fig:msq_seqdia}.
	\item Afsenderen afventer svar fra modtageren før den kan forsætte operation.
\end{itemize}

\begin{figure}[h]
	\centering
	\includegraphics[width=0.6\linewidth]{figs/spm3/msq_seqdia}
	\caption[Syntax for sekvensdiagram med syn -og asynkron msq.]{Sekvensdiagram for syntax ved brug af synkron og asynkron besked køer.}
	\label{fig:msq_seqdia}
\end{figure}

\subsubsection{Its design and implementation}
\textit{MessageQueue} implementering med psudokode:
\todo{unfuck code}
\begin{lstlisting}[morekeywords={unlock, lock, signal, cond_wait}]
// send() function
lock(mtx);

while(num_of_items_in_queue == max_size)
	cond_wait(cond, mtx);

put_item_in_queue(msg, id);

unlock(mtx);
signal(cond);
\end{lstlisting}

\begin{lstlisting}[morekeywords={unlock, lock, signal, cond_wait}]
// receive() function
lock(mtx);

while(num_of_items_in_queue == 0)
	cond_wait(cond, mtx);

item = get_item_from_queue();

unlock(mtx);
signal(cond);

id = item->id;
msg = item->msg;

delete item;
return msg;
\end{lstlisting}

\subsubsection{Impact on design/implementation between before and after the Message Queue}

Brugen af vores MessageQueue har betydet at vi udenfor køens implementering ikke har skulle bekymre os om at huske mutexes, conditionals etc. Man kan sige at vores MessageQueue tilbyder et ekstra abstraktionslag.

\subsection{Event Driven Programming}

\subsubsection{Basic idea}
Event drevet programmering er opgjort af event producenter og event subscribers.
Tråde reagerer på indkommende events, og sover når der ikke sker noget.
Event drevet programmeret er specielt egnet til GUI applikationer hvor user inputs ofte bruges. \\

Der er flere måder at implementere EDP på. I vores ISU øvelse har vi brugt en MessageQueue hvis funktionalitet er beskrevet ovenfor.
Hvis vi ser på figur~\ref{fig:handlPat} vil vores MessageQueue være mellemledet før dispatcheren. Dette er smart da det ikke er sikkert at \textit{dispatcher} kan nå at videresende alle events til deres respektive handlers. MessageQueue'en fungerer derfor som en buffer.
Implementering kommer derfor til at følge dette mønster:
\begin{itemize}
	\item Genereing af beskeden.
	\item Get message fra MsgQueue.
	\item Kør dispatch.
	\item Handle event.
	\item De-alloker besked.
\end{itemize}

\begin{figure}[h]
	\centering
	\includegraphics[width=0.6\linewidth]{figs/spm3/handlersPattern}
	\caption{Illustration af et handler pattern.}
	\label{fig:handlPat}
\end{figure}

\subsubsection{Design - e.g. from sequence diagrams to code (or vice versa)}
Vis evt. PLCS sekvensdiagram