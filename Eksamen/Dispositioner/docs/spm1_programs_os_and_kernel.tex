\section{Programs in relation to the OS and the kernel}

\subsection{Sub topics}

\begin{itemize}
	\item Processes and threads.
	\item Threading model.
	\item Process anatomy.
	\item Virtual memory.
	\item Threads being executed on CPU, the associated scheduler and cache.
\end{itemize}

\subsection{Curriculum}

\begin{itemize}
	\item Slides "Intro to OS's".
	\item Slides "Parallel programs, processes and threads".
	\item OLA: "Anatomy of a program in memory", Gustavo Duarte.
	\item OLA: "The free lunch is over".
	\item OLA: "Virtual memory", pages 131-141.
	\item OLA: " Introduction to operating systems".
	\item OLA: "Multithreading".
	\item Kerrisk: Ch. 3-3.4 - System programming concepts.
	\item Kerrisk: Ch. 29 - Threads: Introduction.
\end{itemize}

\subsection{Exercises}

\begin{itemize}
	\item Posix Threads.
\end{itemize}

\newpage

\subsection{Processes and threads}
\begin{itemize}
	\item En \textbf{process} er en instans af et program, som eksekveres.
	\item En \textbf{thread} er en del af eksekveringen, alle processer har mindst én thread.
\end{itemize}

\subsubsection{Processes}
\begin{itemize}
	\item Har hver sit memory space.
	\item Process A kan ikke skrive i Process B's hukommelse.
	\item Kan kun kommunikere gennem IPC\footnote{Inter-Process Communication, mekanismer kontrolleret af OS.}
	\item Kan skabe andre processer som kan eksekvere det samme eller andre programmer.
\end{itemize}

\subsubsection{Threads}
\begin{itemize}
	\item Alle tråde i en process deler hukommelse på heap'en.
	\item Alle tråde har hver sin stack og program counter. (Tæller instruktioner så CPU ved hvor i koden vi er kommet til).
	\item Tråde er \textit{ikke} individuelle, som processer er, og deler derfor deres kode, data og ressourcer med hinanden, internt i processen. Da hver process' memory er virtuelt og ejet af processen, kan der ikke pilles heri udefra. Der kan med undtagelse kommunikeres med IPC.
	\begin{itemize}
		\item Skal passe på at man ikke sletter de øvrige trådes data.
	\end{itemize}
\end{itemize}

Tråde er forskellige fra processer selvom de deler flere egenskaber og kendetegn. En tråd eksekveres i en process. Man kan sige at en tråd er en enkelt sekvensstrøm inde i en process.

Tråde gør det muligt at eksekvere flere sekvensstrømme ad gangen, og er derved en måde af effektivisere i form af parallelisering. OS's kernel giver gennem system calls mulighed for at oprette og nedlægge tråde.

\paragraph{Thread states}
\begin{itemize}
	\item Running
	\item Blocked (Når tråden ikke vil have CPU time)
	\item Ready (Når tråden gerne vil have CPU time)
	\item Terminated
\end{itemize}


\subsection{Threading model}
Der findes tre forskellige modeller:

\begin{itemize}
	\item User level threading.
	\item Kernel level threading.
	\item Hybrid level threading.
\end{itemize}

\subsubsection{User level threading}
\begin{itemize}
	\item Simpel implementering, ingen kernel support for threads.
	\item Ekstremt hurtig thread kontekst skift ikke brug for kernel handling).
	\item Ikke muligt at håndtere flere CPU-core.
\end{itemize}

\begin{figure}[H]
	\centering
	\includegraphics[width=0.5\linewidth]{figs/spm1/userthreads}
	\caption{User level threading illusteret.}
	\label{fig:userthreads}
\end{figure}

\subsubsection{Kernel level threading}
\begin{itemize}
	\item OS kernel er bevidst om trådene. Giver overhead - er 100 gange langsommere end user level threads.
	\item Mapper direkte til "fysiske" threads som \textit{scheduleren} kan kontrollere.
	\item Effektiv brug af flere kerner.
\end{itemize}

\begin{figure}[H]
	\centering
	\includegraphics[width=0.5\linewidth]{figs/spm1/kernelthreads}
	\caption{Kernel level threading illusteret.}
	\label{fig:kernelthreads}
\end{figure}

\subsubsection{Hybrid level threading}
\begin{itemize}
	\item Komplex implementering.
	\item Kræver god koordination mellem userspace og kernelspace scheduleren - ellers ikke optimal brug af resources.
\end{itemize}

\begin{figure}[H]
	\centering
	\includegraphics[width=0.5\linewidth]{figs/spm1/hybridthreads}
	\caption{Hybrid level threading illusteret.}
	\label{fig:hybridthreads}
\end{figure}

\subsection{Process anatomy}
\begin{itemize}
	\item Når et program startes, starter en ny process.
	\item En process kører i sin egen memory sandbox, som et \textit{virtual address space} (ca. 4GB på 32-bit platform).
	\item Hver process har sin egen \textbf{pagetable/virtual address space}.
	\item Den virtuelle memory mapper til fysisk memory addresser vha. pagetables.
	\item Alle processer har \textbf{virtual address space}, hvor en del er bestemt til kernel space.
	\item Kernel space er ens for alle processor og mapper til samme fysiske hukommelse.
	\item Kernel space er flagged i pagetable med privileged code, så kun kernel space programmer kan tilgå det memory.Der forekommer Page fault hvis en user-space process ulovigt forsøger at tilgå dette hukommelse.
\end{itemize}

\begin{figure}[h]
	\centering
	\includegraphics[width=0.7\linewidth]{figs/spm1/pagetable}
	\caption{Virtual mapping via pagetable.}
	\label{fig:pagetable}
\end{figure}

\begin{figure}[H]
	\centering
	\includegraphics[width=\linewidth]{figs/spm1/memorydiagram}
	\caption{Diagram over standard segment layout.}
	\label{fig:memorydiagram}
\end{figure}

Når der mappes virtuel memory til processer er der af sikkerhedshensyn indført random offsets for startaddressen på det virtuelle memory.\\

\begin{figure}
	\centering
	\includegraphics[width=0.7\linewidth]{figs/spm1/virtualaddressspace}
	\caption{Typical memory layout of a process on Linux/x86-32.}
	\label{fig:virtualaddressspace}
\end{figure}


Resten, udover kernel space processens egen.\\

Her findes: Stack, heap, memory mapping, BSS, data og text/code segment.

\begin{figure}[H]
	\centering
	\includegraphics[width=\linewidth]{figs/spm1/memoryspace}
	\caption{Diagram for process switch.}
	\label{fig:memoryspace}
\end{figure}

Alle processer har deres eget virtual address space, som der skiftes mellem ved context switches, se figur~\ref{fig:memoryspace}.

\paragraph{BSS}
Indeholder \textbf{ikke} initialiserede statiske variabler. Dette område er anonymt (mapper ikke til nogen fil, men er blok "reseveret").

\paragraph{Data segment}
Indeholder statisk allokerede og initialiserede variabler. Dette område er \textbf{ikke} anonymt og mapper til en fil på disken. Data segmentet kortlægger de initialiserede statiske værdier der er givet i source koden. Da dette er privat bliver ændringer ikke gemt i det mappede område.

\paragraph{Text segment}
Tag et eksempel: indholdet af en pointer (X-bit addresse) er i data segmentet men selve det den peger på ligger i \textbf{text segmentet}, som er \textit{read-only} og indeholder alt din kode samt literals. Text segmentet mapper ens binære filer i hukommelsen (dette er read only, og forsøg på adgang resulterer i Segmentation fault).

\subsection{Virtual memory}
Linux processer bliver eksekveret i et virtuelt miljø. På denne måde tror hver process at den har al hukommelsen for sig selv.\\

Vigtige grunde til at vi bruger virtuel hukommelse:

\begin{itemize}
	\item Resource virtualization.
	\begin{itemize}
		\item En process skal ikke tænke på hvor meget hukommelse der er tilgængeligt.
		\item Virtuel hukommelse gør en begrænset mængde fysisk hukommelse til en stor ressource.
	\end{itemize}
	\item Information isolation.
	\begin{itemize}
		\item Hver process arbejder i sit eget miljø, derved kan den ikke læse (eller skrive i) en anden process's hukommelse.
		\item Forbedre sikkerheden, da en process således ikke kan ''spionere'' på en anden process.
	\end{itemize}
	\item Fault isolation.
	\begin{itemize}
		\item Kan ikke fucke andre processer op, da den ikke har adgang der deres hukommelse.
		\item Hvis en process crasher/fejler/etc. ødelægger det ikke resten af systemet - problemet er isoleret i processen.
	\end{itemize}
\end{itemize}

\subsection{Threads being executed on CPU, associated Scheduler and Cache}\label{sec:execpu}
\textbf{Scheduleren} sætter processer op til eksekvering på CPU og sørger for at skifte (switche) mellem processer:

\begin{enumerate}
	\item Interrupt.
	\item Save context.
	\item Restore context.
	\item Resume execution.
\end{enumerate}

Der er to måder at schedule på:

\begin{itemize}
	\item Preemptive - Preemptive modtager interrupt regulært og skifter proces.
	\item Non-preemptive - Non-preemptive kræver at processen selv siger ''Nu må der skiftes''.
\end{itemize}

\subsubsection{Cache}
De fleste CPU'er har minimum 3 typer caches:
\begin{itemize}
	\item Instruction cache - fremskynder instruction fetching.
	\item Data cache - fremskynder fetch and store, og deles op i cache levels (L1, L2 osv.).
	\item TLB (Translation Lookaside Buffer) - fremskynder oversættelse af virtuel til fysiske addresser (TLB er en del af MMU'en\footnote{Memory Management Unit})
\end{itemize}
TLB cache bruges hver gang en virtuel til fysisk oversættelse sker. De seneste data gemmes da i TLB’en. Dette forøger hastigheden på næste lookup til samme addresse.